\documentclass{dreamClass}

\title{\heiti\textbf{某门课 \quad 某次作业}}
\author{123456789 赵志辉\thanks{\href{mailto:test@example.com}{test@example.com}}}
\affil{九乡河文理学院}
\date{\today}

\begin{document}
\maketitle
\thispagestyle{empty}

\section{喵}
下面是带圆圈的编号列表:
\begin{enumerate}[label=\large\protect\textcircled{\small\arabic*}]
    \item 喵
    \item 喵
\end{enumerate}

\autoref{pic:desktop}是一幅插图。
\begin{figure}
    \centering
    \includegraphics[width = 0.5\textwidth]{screenshot.jpg}
    \caption{一张截图}\label{pic:desktop}
\end{figure}

\autoref{pic:flag}是\(2 \times 2\) 摆放的四张图片。
\begin{figure}
    \centering
    \begin{minipage}{.5\textwidth}
        \centering
        \includegraphics[width=\textwidth]{pic/flag/flag_500kb.png}
        \caption{500KB}
    \end{minipage}%
    \begin{minipage}{.5\textwidth}
        \centering
        \includegraphics[width=\textwidth]{pic/flag/flag_200kb.png}
        \caption{200KB}
    \end{minipage}

    \begin{minipage}{.5\textwidth}
        \centering
        \includegraphics[width=\textwidth]{pic/flag/flag_100kb.png}
        \caption{100KB}
    \end{minipage}%
    \begin{minipage}{.5\textwidth}
        \centering
        \includegraphics[width=\textwidth]{pic/flag/flag_50kb.png}
        \caption{50KB}
    \end{minipage}
    \caption{相机拍摄的FLIF格式的图片在不同位置截断后的效果\label{pic:flag}}
\end{figure}


\autoref{code:junk}是作为浮动体的代码块:
\begin{listing}
    \caption{凑字数用的代码\label{code:junk}}
    \begin{minted}{cpp}
    int main() {}
    \end{minted}
\end{listing}

\autoref{code:junk-again}是并非浮动体的代码块:
\begin{codeblock}
    \caption{复读机\label{code:junk-again}}
    \begin{minted}{cpp}
    int main() {}
    \end{minted}
\end{codeblock}

这是行内的代码片段:\mintinline{cpp}{wstring price = L"九磅十五便士"}。

\autoref{code:engine}是使用文件作为代码块的内容:
\begin{codeblock}
    \caption{\texttt{Engine.hpp}的内容\label{code:engine}}
    \inputminted{cpp}{../code/engine/Engine.hpp}
\end{codeblock}

下面是一些数学式:
\[
    \begin{align*}
        P(x) & = \int_{0}^{x} \beta e^{ - \beta x} \dif x \\
             & = \int_{0}^{\beta x} e^{ - t} \dif t       \\
             & = - e^{ - u} \Big|_{0}^{\beta x}           \\
             & = 1 - e^{ - \beta x}\text{。}
    \end{align*}
\]
\[
    \begin{aligned}
        \begin{array}{cl}
            \text{maximize} & H = \bm{p}^T\bm{q} \\
            \text{ s.t. }   & \bm{p}^T\bm{1} - 1 = 0\text{。}
        \end{array}
    \end{aligned}
\]
考虑\(\mathbf{H}_{i j}=\frac{\partial^{2} f}{\partial x_{i} \partial x_{j}}\)。
又\(\operatorname{Pr}(X \geq 10^3 + 2000 | X \geq 2000) = \operatorname{Pr}(X \geq 10^3)\),

所以\(I * F = \begin{bmatrix}
    2 & -2  & 1 & - 1 \\
    4 & - 5 & 4 & - 3 \\
    2 & - 3 & 3 & - 2
\end{bmatrix}\)

下面是一个\texttt{minipage}的示例,得到\autoref{pic:pr}和\autoref{pic:roc}。
\begin{figure}
    \centering
    \begin{minipage}{.5\textwidth}
        \centering
        \includegraphics[width = \textwidth]{pr.pdf}
        \caption{P-R曲线\label{pic:pr}}
    \end{minipage}%
    \begin{minipage}{.5\textwidth}
        \centering
        \includegraphics[width = \textwidth]{roc.pdf}
        \caption{ROC曲线\label{pic:roc}}
    \end{minipage}
\end{figure}

本证明受\href{https://datawhalechina.github.io/pumpkin-book/#/chapter2/chapter2?id=221}{这个网页}启发。
\end{document}
