\documentclass{dreamClass}
\usepackage{siunitx} % Provides the \SI{}{} and \si{} command for typesetting SI units

\title{
    \vspace{-50pt}
    \textbf{\Huge《大数据综合处理》课程设计}\\
    \textbf{\huge 金庸的江湖}
}

\author[1]{123456789赵志辉\thanks{\href{mailto:zhh.zhao@outlook.com}{zhh.zhao@outlook.com}}}
\author[1]{123456790张三}
\author[1]{123456791李四}
\author[2]{233233233喵喵喵}
\affil[1]{计算机科学与技术系,九乡河文理学院}
\affil[2]{喵喵喵,B612星球}
\date{\today}

\begin{document}
\maketitle
\thispagestyle{empty}

\section{插图}\label{sec:1}
\subsection{一张插图}\label{sec:2}
\cref{pic:desktop}是一幅插图。\cref{sec:1,sec:2}是两节。
\begin{figure}[H]
    \centering
    \includegraphics[width = 0.5\textwidth]{screenshot.jpg}
    \caption{一张截图}\label{pic:desktop}
\end{figure}
\subsection{并排摆放的插图}
\cref{pic:flag}是\(2 \times 2\) 摆放的四张图片。
\begin{figure}
    \centering
    \begin{subfigure}{.5\textwidth}
        \centering
        \includegraphics[width=\textwidth]{screenshot.jpg}
        \caption{500KB}
    \end{subfigure}%
    \begin{subfigure}{.5\textwidth}
        \centering
        \includegraphics[width=\textwidth]{screenshot.jpg}
        \caption{200KB}
    \end{subfigure}

    \begin{subfigure}{.5\textwidth}
        \centering
        \includegraphics[width=\textwidth]{screenshot.jpg}
        \caption{100KB}
    \end{subfigure}%
    \begin{subfigure}{.5\textwidth}
        \centering
        \includegraphics[width=\textwidth]{screenshot.jpg}
        \caption{50KB}
    \end{subfigure}
    \caption{相机拍摄的FLIF格式的图片在不同位置截断后的效果\label{pic:flag}}
\end{figure}

\cref{tbl:1}是一个\texttt{subtable}的示例,得到\cref{tbl:1-1}和\cref{tbl:1-2}。
\begin{table}
    \caption{修正后的$\kappa_t$数据记录表}\label{tbl:1}
    \begin{subtable}{0.45\textwidth}
        \centering
        \caption{298.2K}\label{tbl:1-1}
        \begin{tabular}{ccc}
            \toprule
            $t$/min & $\kappa_t$/\si{\micro S.cm^{-1}} & $\frac{\kappa_{0}-\kappa_{t}}{t}$/\si{\micro S.cm^{-1}.min^{-1}} \\
            \midrule
            6       & 1892                             & 49.7                                                             \\
            9       & 1793                             & 44.1                                                             \\
            12      & 1712                             & 39.8                                                             \\
            15      & 1648                             & 36.1                                                             \\
            20      & 1559                             & 31.6                                                             \\
            25      & 1492                             & 27.9                                                             \\
            30      & 1439                             & 25.0                                                             \\
            40      & 1361                             & 20.7                                                             \\
            50      & 1305                             & 17.7                                                             \\
            60      & 1260                             & 15.5                                                             \\
            \bottomrule
        \end{tabular}
    \end{subtable}
    \hfill
    \begin{subtable}{0.45\textwidth}
        \centering
        \caption{308.2K}\label{tbl:1-2}
        \begin{tabular}{ccc}
            \toprule
            $t$/min & $\kappa_t$/\si{\micro S.cm^{-1}} & $\frac{\kappa_{0}-\kappa_{t}}{t}$/\si{\micro S.cm^{-1}.min^{-1}} \\
            \midrule
            4       & 1788                             & 73.0                                                             \\
            6       & 1692                             & 64.7                                                             \\
            8       & 1621                             & 57.4                                                             \\
            10      & 1560                             & 52.0                                                             \\
            12      & 1512                             & 47.3                                                             \\
            15      & 1455                             & 41.7                                                             \\
            18      & 1405                             & 37.5                                                             \\
            21      & 1368                             & 33.9                                                             \\
            24      & 1336                             & 31.0                                                             \\
            27      & 1312                             & 28.4                                                             \\
            30      & 1288                             & 26.4                                                             \\
            \bottomrule
        \end{tabular}
    \end{subtable}
\end{table}
\subsection{环绕式插图}
\begin{wrapfigure}{r}{.3\textwidth}
    \centering
    \includegraphics[width=.28\textwidth]{screenshot.jpg}
    \caption{双管式电导池示意图}\label{pic:wow}
\end{wrapfigure}
DDS—11A(T)型电导率仪(附DIS型铂黑电极)1台;计时器1只;恒温槽1套;双管式电导池2只(见\cref{pic:wow});胖肚移液管(25mL)3支;烧杯(50mL)1只;容量瓶(250mL)一只;称量瓶(Φ25mm×23mm)1只。

\section{代码}

\cref{code:junk}是代码块:
\begin{listing}
    \caption{凑字数用的代码\label{code:junk}}
    \begin{minted}{cpp}
    int main() {}
    \end{minted}
\end{listing}

\cref{code:junk-again}也是代码块:
\begin{listing}[H]
    \caption{复读机\label{code:junk-again}}
    \begin{minted}[linenos]{cpp}
    int main(void) {
        // 我们都知道$E = mc^2$
        int i = 42;
        return 0;
    }
    \end{minted}
\end{listing}

\subsection{行内代码}
这是行内的代码片段:\mintinline{cpp}{wstring price = L"九磅十五便士"}。
\subsection{插入文件作为代码}
\cref{code:engine}是使用文件作为代码块的内容:
\begin{listing}
    \caption{\texttt{Engine.hpp}的内容\label{code:engine}}
    \inputminted{cpp}{code/test.cpp}
\end{listing}

\section{公式}
\subsection{独立成行的公式}
下面是一些数学式:
\[
    \begin{aligned}
        P(x) & = \int_{0}^{x} \beta e^{ - \beta x} \dif x \\
             & = \int_{0}^{\beta x} e^{ - t} \dif t       \\
             & = - e^{ - u} \Big|_{0}^{\beta x}           \\
             & = 1 - e^{ - \beta x}\text{。}
    \end{aligned}
\]
\[
    \begin{aligned}
        \begin{array}{cl}
            \text{maximize} & H = \bm{p}^T\bm{q}              \\
            \text{ s.t. }   & \bm{p}^T\bm{1} - 1 = 0\text{。}
        \end{array}
    \end{aligned}
\]

\begin{equation}\label{eq:1}
    E = mc^2
\end{equation}
\cref{eq:1}看起来就很厉害的样子。

\subsection{行内公式}
考虑\(\mathbf{H}_{i j}=\frac{\partial^{2} f}{\partial x_{i} \partial x_{j}}\)。
又\(\operatorname{Pr}(X \geq 10^3 + 2000 | X \geq 2000) = \operatorname{Pr}(X \geq 10^3)\),

所以\(I * F = \begin{bmatrix}
    2 & -2  & 1 & - 1 \\
    4 & - 5 & 4 & - 3 \\
    2 & - 3 & 3 & - 2
\end{bmatrix}\)

本证明受\href{https://datawhalechina.github.io/pumpkin-book/#/chapter2/chapter2?id=221}{这个网页}启发。

\section{定理}
\begin{theorem}\label{thm:1}
    喵喵喵
\end{theorem}

首先证明\cref{lemma:1}。
\begin{lemma}\label{lemma:1}
    喵
\end{lemma}

下面证明\cref{thm:1}:
\begin{proof}
    喵喵喵喵喵喵喵
\end{proof}


\section{算法}
\begin{problem}(习题13.1)
修改Floyd-Warshall算法,使其在给出所有点对之间的最短距离的同时,分别给出所有点对之间的:
\begin{enumerate}
    \item \emph{后继路由表},定义为矩阵\texttt{GO}。\texttt{GO[i][j] = k},表示从\texttt{i}到\texttt{j}的下一跳为\texttt{k}。
    \item \emph{前驱路由表},定义为矩阵\texttt{FROM}。\texttt{FROM[i][j] = k},表示从\texttt{i}到\texttt{j}的最后一跳为\texttt{k}。
\end{enumerate}
\end{problem}

\subparagraph*{第1问}
注意到,在Floyd-Warshall算法的进行途中,
矩阵\texttt{D}中始终存放着当前的子问题下的最短路径,最终结束时子问题的最短路径则变为原问题的最短路径。
所以我们应当在\texttt{D}被更新的时候同时更新路由表。
算法如\cref{alg:floyd-route-1}所示。

\begin{algorithm}
    \caption{给出后继路由表的Floyd-Warshall算法}\label{alg:floyd-route-1}
    \Indm
    \KwData{图G的邻接矩阵D,空的矩阵GO}
    \KwResult{G的后继路由表GO}
    \Indp

    \ForAll{ i, j使得D[i][j] $\not=\infty$}{
    GO[i][j] = j\tcp*{初始化路由表}\label{algline:floyd-route-init}
    }
    \For{ k = 0 \KwTo n} {
        \For{ i = 0 \KwTo n} {
            \For{ j = 0 \KwTo n} {
                \If{ D[i][k] + D[k][j] < D[i][j]}{
                    D[i][j] = D[i][k] + D[k][j]\;
                    GO[i][j] = GO[i][k]\tcp*{发现更短路径,更新路由表}\label{algline:floyd-route-proceed}
                }
            }
        }
    }
\end{algorithm}
\subparagraph*{第2问}
如果要求前驱路由表而不是后继,我们只需简单地把\cref{alg:floyd-route-1}的\cref{algline:floyd-route-init}改成
\begin{listing}[H]
    \begin{minted}[linenos=false]{cpp}
        GO[i][j] = i
    \end{minted}
\end{listing}
再把\cref{algline:floyd-route-proceed}改为
\begin{listing}[H]
    \begin{minted}[linenos=false]{cpp}
        GO[i][j] = GO[k][j]
    \end{minted}
\end{listing}
即可。
这样,\texttt{GO}给出的将是从\texttt{i}到\texttt{j}的最短路径的\emph{最后一跳}。

\section{其他}\label{sec:other}
\cref{sec:other}是最后一节。
\subsection{带圆圈的编号列表}
\begin{enumerate}[label=\large\protect\textcircled{\small\arabic*}]
    \item 喵
    \item 呜
\end{enumerate}

\text{\kaishu 正常 \textbf{粗体} \textit{斜体}。}

\text{\songti 正常 \textbf{粗体} \textit{斜体}。}

\text{\heiti 正常 \textbf{粗体} \textit{斜体}。}

\text{\fangsong 正常 \textbf{粗体} \textit{斜体}。}

\begin{remark}
    不过绝大多数用\LaTeX{}进行学术论文写作的同学,在新建文件不久之后就会认识到一个关键的事实,有诗云:
    \begin{enumerate}
        \item 浮动体遍地乱跑,标签溢出边框
        \item 多一字\verb+\hbox badness 10000+,少一字文档编译失败
        \item 看似应有尽有,实际啥都有bug
    \end{enumerate}
    甚至于说给文本加下划线这么简单的功能,为了能和lualatex兼容,都要先…再…,其积弊如此。

    当然了!这个时候自然会天降神兵,说,“我来救你!”

    这个救星就是Mircosoft\textsuperscript{\small\textregistered} Word!
\end{remark}

\end{document}
