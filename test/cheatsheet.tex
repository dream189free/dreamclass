\documentclass{dreamClass}

\title{\heiti\textbf{某门课 \quad 某次作业}}
\author{123456789 \quad 赵志辉\thanks{\href{mailto:test@example.com}{test@example.com}}}
\affil{九乡河文理学院}
\date{\today}

\begin{document}
\maketitle
\thispagestyle{empty}

\section{听力测试}
下面开始听力测试,请同学们做好准备。
% 带圆圈的编号列表
\begin{enumerate}[label=\large\protect\textcircled{\small\arabic*}]
    \item First item
    \item Second item
    \item Third item
    \item Fourth item
\end{enumerate}

\subsection{短对话}
\begin{figure}
    \centering
    \includegraphics[width = 0.5\textwidth]{screenshot.jpg}
    \caption{一张截图}\label{pic:desktop}
\end{figure}
听下面五段对话,每段对话后有一个小题,从题中所给的A、B、C三个选项中,选出最佳选项。听每段对话前,你将有五秒钟的时间阅读各小题的内容。每段对话读一遍。
现在,你有五秒钟的时间看试卷上的例题如\autoref{pic:desktop}。
\subsection{例题}
例如,你将听到以下内容:
% 作为浮动体的代码块
\begin{listing}
    \caption{凑字数用的代码\label{code:junk}}
    \begin{minted}{cpp}
    int main() {}
    \end{minted}
\end{listing}
% 行内的代码片段
衬衫的价格\mintinline{cpp}{wstring price = L"九磅十五便士"}。

\subsection{听力考试到此结束}
现在,你有两分钟的时间,将试卷上的答案转涂到客观题答题卡上。
% 并非浮动体的代码块
\begin{codeblock}
    \caption{复读机\label{code:junk-again}}
    \begin{minted}{cpp}
    int main() {}
    \end{minted}
\end{codeblock}

% 使用文件作为代码块的内容
\begin{codeblock}
    \caption{\texttt{Engine.hpp}的内容}
    \inputminted{cpp}{../code/engine/Engine.hpp}
\end{codeblock}

\end{document}
